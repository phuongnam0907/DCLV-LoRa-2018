\renewcommand{\bibname}{Tài liệu tham khảo}
\begin{thebibliography}{10}
\addcontentsline{toc}{chapter}{Tài liệu tham khảo}
\baselineskip
21pt

%
%Tên tác giả (năm). Tên tài liệu [online], ngày tháng năm truy cập nguồn thông tin, từ <URL>.
%


\bibitem{tl1} 
DTT. Internet of things là gì? [online], viewed 25 September 2018, from: <\href{http://iot.dtt.vn/InternetofThings.html}{http://iot.dtt.vn/InternetofThings.html}>.


\bibitem{tl2}
Ho Nguyen (May 2018). Định nghĩa khái niệm Internet of Things (IoT) - Internet Vạn Vật [online], viewed 25 September 2018, from: <\href{https://blog.trginternational.com/vi/dinh-nghia-khai-niem-internet-of-things-iot-internet-van-vat}{https://blog.trginternational.com/vi/dinh-nghia-khai-niem-internet-of-things-iot-internet-van-vat}>.

\bibitem{tl3}
Tỷ Phú (April 2018). Android Studio là gì? [online], viewed 1 October 2018, from: <\href{https://quantrimang.com/android-studio-la-gi-149713}{https://quantrimang.com/android-studio-la-gi-149713}>.

\bibitem{tl4}
Eric Lamarre, McKinsey \& Company (May 2017). Making sense of Internet of Things platforms [online], viewed 1 October 2018, from: <\href{https://www.mckinsey.com/business-functions/digital-mckinsey/our-insights/making-sense-of-internet-of-things-platforms}{https://www.mckinsey.com/business-functions/digital-mckinsey/our-insights/making-sense-of-internet-of-things-platforms}>.

\bibitem{tl5}
Thạch An (August 2016). Những hệ điều hành cho IoT trong tương lai [online], viewed 1 October 2018, from: <\href{http://www.pcworld.com.vn/articles/cong-nghe/cong-nghe/2016/08/1249319/nhung-he-dieu-hanh-danh-cho-iot-trong-tuong-lai/}{http://www.pcworld.com.vn/articles/cong-nghe/cong-nghe/2016/08/1249319/nhung-he-dieu-hanh-danh-cho-iot-trong-tuong-lai/}>.

\bibitem{tl6}
Bùi Danh Nam (May 2018). Giới thiệu về Android Things 1.0 [online], viewed 1 October 2018, from: <\href{https://viblo.asia/p/gioi-thieu-ve-android-things-10-maGK7M4xlj2}{https://viblo.asia/p/gioi-thieu-ve-android-things-10-maGK7M4xlj2}>.

\bibitem{tl7}
Lưu Hoàng Trúc (December 2016). [Android Things] Phần 1: Tổng quan về IoT và AT - Cài đặt môi trường cho Android Things với kit Raspberry Pi 3 [online], viewed 2 October 2018, from: <\href{https://viblo.asia/p/android-things-phan-1-tong-quan-ve-iot-va-at-cai-dat-moi-truong-cho-android-things-voi-kit-raspberry-pi-3-jvlKaqqDKVr}{https://viblo.asia/p/android-things-phan-1-tong-quan-ve-iot-va-at-cai-dat-moi-truong-cho-android-things-voi-kit-raspberry-pi-3-jvlKaqqDKVr}>.

\bibitem{tl8}
Paul Trebilcox-Ruiz (January 2017), Vietnamese translation by Dai Phong. Giới thiệu về Android Things [online], viewed 2 October 2018, from: <\href{https://code.tutsplus.com/vi/articles/introduction-to-android-things--cms-27892}{https://code.tutsplus.com/vi/articles/introduction-to-android-things--cms-27892}>.

\bibitem{tl9}
Duy Luân (February 2018). Discussion in "Thông tin công nghệ": Cách mạng công nghiệp 4.0 là gì, nó ảnh hưởng tới bản thân các bạn ra sao? [online], viewed 5 October 2018, from: <\href{https://tinhte.vn/threads/cach-mang-cong-nghiep-4-0-la-gi-no-anh-huong-toi-ban-than-cac-ban-ra-sao.2770055/}{https://tinhte.vn/threads/cach-mang-cong-nghiep-4-0-la-gi-no-anh-huong-toi-ban-than-cac-ban-ra-sao.2770055/}>.

\bibitem{tl10}
Trung Kiên (April 2015). Chuẩn giao tiếp SPI [online], viewed 10 October 2018, from: <\href{https://kienltb.wordpress.com/2015/04/05/chuan-giao-tiep-spi/}{https://kienltb.wordpress.com/2015/04/05/chuan-giao-tiep-spi/}>.

\bibitem{tl11}
Ý Tưởng Nhanh (2017). Giao tiếp SPI với vi điều khiển PIC (Phần 1) [online], viewed 12 October 2018, from: <\href{http://ytuongnhanh.vn/chi-tiet/giao-tiep-spi-voi-vi-dieu-khien-pic-phan-1-144.html}{http://ytuongnhanh.vn/chi-tiet/giao-tiep-spi-voi-vi-dieu-khien-pic-phan-1-144.html}>.

\bibitem{tl12}
AloChip. CÔNG NGHỆ TRUYỀN THÔNG KHÔNG DÂY LORA [online], viewed 15 October 2018, from: <\href{https://alochip.com/blog/cong-nghe-truyen-thong-khong-day-lora\_8.html}{https://alochip.com/blog/cong-nghe-truyen-thong-khong-day-lora\_8.html}>.

\bibitem{tl13}
BKAII. 10 ứng dụng thế giới thực của Internet of Things [online], viewed 1 November 2018, from: <\href{https://bkaii.com.vn/tin-tuc/222-10-ung-dung-the-gioi-thuc-cua-internet-of-things}{https://bkaii.com.vn/tin-tuc/222-10-ung-dung-the-gioi-thuc-cua-internet-of-things}>.

\bibitem{tl14}
LoRa Alliance™. About the LoRaWAN™ Specification [online], viewed 20 November 2018, from: <\href{https://lora-alliance.org/lorawan-for-developers}{https://lora-alliance.org/lorawan-for-developers}>.

\bibitem{tl15}
MAKER.IO STAFF (October 2016). Introduction to LoRa Technology – The Game Changer [online], viewed 24 November 2018, from: <\href{https://www.digikey.com/en/maker/blogs/introduction-to-lora-technology}{https://www.digikey.com/en/maker/blogs/introduction-to-lora-technology}>.

\bibitem{tl16}
Hunghv (October 2017). Arduino là gì? Lập trình Arduino bằng C/C++ [online], viewed 18 November 2018, from: <\href{http://hanhtranglaptrinh.vn6.vn/arduino-la-gi-lap-trinh-arduino-bang-c-c-plus-plus/}{http://hanhtranglaptrinh.vn6.vn/arduino-la-gi-lap-trinh-arduino-bang-c-c-plus-plus/}>.

\bibitem{tl17}
Semtech Corporation (July 2013). SX1272/3/6/7/8: LoRa Modem - Designer’s Guide [online], viewed 2 November 2018, from: <\href{https://www.semtech.com/uploads/documents/LoraDesignGuide\_STD.pdf}{https://www.semtech.com/uploads/documents/LoraDesignGuide \_STD.pdf}>.

\bibitem{tl18}
Jhon\_Control (2017). Introduction LoRa \& Module RFM95/RFM95W Hoperf [online], viewed 7 December 2018, from: <\href{https://www.instructables.com/id/Introduction-LoRa-Module-RFM95-RFM95W-Hoperf/}{https://www.instructables.com/id/Introduction-LoRa-Module-RFM95-RFM95W-Hoperf/}>.

\end{thebibliography}
